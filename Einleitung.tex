\section{Einleitung}



Bereits im 17. Jahrhunder beschrieb Huygens, der Begründer der Wellenoptik, dass Licht in seiner Ausbreitung Wellencharakter zeigt. Dem entgegen stellte sich die von Isaac Newton begründete Theorie Licht als Strahl von Teilchen zu betrachten. Es folgten viele Jahren von Diskussionen und Experimente der konkurrierenden theoretischen Lager, bis schließlich am Anfang des 20. Jahrhundert der Welle-Teilchen-Dualismus begründet wurde. Dieser besagt, dass Quantenobjekte, also auch Lichtteilchen, sowohl als Welle wie auch Teilchen beschreiben werden können. Der in diesem Versuch verwendete Effekt der Beugung an einem Einfachspalt zur Bestimmung einer unbekannten Wellenlänge ist ein sehr frühes Beispiel eines aus der Wellenoptik kommenden Experimentes.
