
\setlength\abovedisplayshortskip{20pt}
\setlength\belowdisplayshortskip{20pt}
\setlength\abovedisplayskip{20pt}
\setlength\belowdisplayskip{20pt}


\section{Theoretische Grundlagen \cite{wedler}}

Trifft kollimiertes Licht auf einem kleinen Spalt der Breite $b$ zeigt es ein Verhalten, welches der in der Strahlenoptik geltenden Gesetze widerspricht. Anzunehmen sei eigentlich, dass paralleles Licht, welches durch den Spalt auf eine dahinter liegende Sammellinse fällt, nur in einem Punkt gebündelt erscheinen sollte. Jedoch beobachtet man stattdessen ein Beugungsmuster, welches auch als Fraunhofer-Beugungsmuster bezeichnet wird.



Dass das Licht ein solches Verhalten zeigt kann auf das Huygens-Prinzip zurückgeführt werden. Dieses besagt, dass jeder Punkt einer ebenen Lichtwelle (eine sogenannte Wellenfront), Ausgangspunkt einer neuen Elementarwelle ist. Trifft nun eine solche Front auf einen Spalt der Breite b wird jeder Punkt der innerhalb des Spalts liegt zum Ausgangspunktes einer Elementarwelle. Es zeigt sogenannte Grenzeffekte. Es breiten sich also Kreisförmig in den Schatten des Raums hinter dem Spalt aus. Dabei zeigen Lichtwellen das zu beobachtenden Interferenzmuster. Die  Intensität des zu beobachtende Interferenzmuster kann durch einen Verlauf einer quadratischen \textit{sinus Cardinalis} ($sinc(x)=\frac{sin(x)}{x}$) Funktion beschrieben werden.


\begin{equation}
 \begin{cases}\quad \quad
  \text{für} \quad \quad   x\neq 0 & I(x)=I_0 \cdot\left(\frac{sin(x)}{x}\right)^2 \quad \quad \text{mit} \quad \quad x =\frac{\pi \cdot b\cdot sin(\varphi)}{\lambda}  \\\\ \quad \quad
  \text{für} \quad \quad    x=0 & I(x)=I_0\\    
   \end{cases}
\end{equation}
Dabei ist $b$ die Spaltbreite $\varphi$ der Beugungswinkel und $\lambda$ die Wellenlänge des Lichts.


\begin{equation}
I(\varphi)=I_0 \cdot\left(\frac{sin(k\cdot sin(\varphi) )}{k\cdot sin(\varphi)}\right)^2 \quad \quad \text{mit} \quad \quad k =\frac{\pi \cdot b}{\lambda }
\end{equation}

Es gilt, dass bei $\varphi=0$ das Inentsitätsmaximum $I_0$ das sogenannte Hauptmaximums gefunden wird und jedes weitere Maxima nach folgender Gesetzmäßigkeit gefunden werden kann:




\begin{equation}
b\cdot sin(\varphi)=(2 \cdot k+1)\cdot \frac{\lambda}{2}=\left(k+\frac{1}{2}\right)\cdot \lambda
\end{equation} 

